% !TeX program = xelatex
% !TeX encoding = UTF-8
%  请使用xeLaTeX进行编译
\documentclass[postdoc,oneside]{sistthesis}
% 选项  postdoc|doctor|master  默认postdoc
%      onesid|twoside         postdoc 请选单栏(编写规则要求单面)


%%% 引用格式使用数字  [1,2] 
%\usepackage[numbers,comma, sort&compress]{natbib}      

%%% 引用格式使用上标  ^{[1,2]}
\usepackage[super,square,comma,sort&compress]{natbib} 

%%%使用GBT7714标准的参考文献格式 (按引用顺序排列参考文献)
\bibliographystyle{gbt-numerical} 

%%%使用GBT7714标准的参考文献格式(按作者-日期排列参考文献)
%\bibliographystyle{gbt-author-year} 

%\usepackage{hypernat} 
\usepackage{subfig}
\usepackage{comment,multirow,float}
% set search path
\graphicspath{{pic/}}

%%%%% 分类号等
\classification{TM13}
\confidential{不涉密}
\UDC{}
\IDnumber{}

%%%%%%%%%%%%%%%%%%%%%%%%%%%%%%%%%%%%%%%%%%%%%%%%%%%%%%%%%%%%%%%%%%%%%%%%%%%%%%%%%%%%%
%%%%%%%%%%%  博士后工作报告将此处注释取消
%%%%%   封面
%\title{新型有机非线性光学材料的探索~\ \\\ ~ -- 从分子到晶体的材料化学过程~}
%\author{方奇}
%\jobbegin{2015年7月22日}
%\jobend{2018年3月31日}
%\reportfinish{2015年7月—2018年3月}
%\submitdate{2018年3月} 
%%
%%%%%%% 题名页
%\englishtitle{THE EXPLORATION FOR NEW ORGANIC NLO\\
%--MATERIALS TIIE MATERIALS CIIEMISTRY PROCESS\\
%--FROM MOLECULES TO CRYSTALS}
%\discipline{电子科学与技术}
%\major{电磁场与微波技术}
%\institute{中国科学院上海微系统与信息技术研究所}
%\address{上海}
%%%%%%%%%%%%%%%%%%%%%%%%%%%%%%%%%%%%%%%%%%%%%%%%%%%%%%%%%%%%%%%%%%%%%%%%%%%%%%%%%%%%%%%%%%%%%%%%



%%%%%%%%%%%%%%%%%%%%%%%%%%%%%%%%%%%%%%%%%%%%%%%%%%%%%%%%%%%%%%%%%%%%%%%%%%%%%%%%%%%%%
%%%%%%%%%%%  学位论文将此处注释取消
%%%%    中文题名页
%\title{题目}
%\setitle{Englishi}
%\setitlee{title}
%\author{sslchi}
%\discipline{电子科学与技术}
%\major{电磁场与微波技术}
%\research{微分方程数值解}
%\advisor{指导教师}
%\school{信息科学与技术学院}
%\submitdate{2018年3月} 

%%%%     英文题名页
%\englishtitle{English\\ title}
%\englishauthor{sslchi}
%\englishschool{School of Information Science and Technology}
%\englishsubmitdate{Apr., 2017}

%%%%%%%%%%%%%%%%%%%%%%%%%%%%%%%%%%%%%%%%%%%%%%%%%%%%%%%%%%%%%%%%%%%%%%%%%%%%%%%%%%%%%%%%%%%%%%




\begin{document}

	
\maketitle
\makeenglishtitle

\frontmatter
%\begin{announce}

\begin{center}
\textbf{\zihao{3}独创性声明}
\end{center}
\hskip 18pt
本人声明。。
\vskip 18pt
\begin{tabular}{cc}
学位论文作者签名:\hskip 80pt\    & 签字日期:\hskip 40pt 年\hskip 20pt 月\hskip 20pt 日\\
\end{tabular}
\vskip 120pt
\begin{center}
\textbf{\zihao{3}学位论文版权使用授权书}
\end{center}
\hskip 18pt
本学位论文作者完全了解。。
(保密的学位论文在解密后适用本授权书)
\vskip 18pt
\begin{tabular}{ll}
学位论文作者签名:\hskip 80pt\    & 导师签名:\hskip 80pt\  \\
签字日期:\hskip 40pt 年\hskip 20pt 月\hskip 20pt 日 &
签字日期:\hskip 40pt 年\hskip 20pt 月\hskip 20pt 日\\
\end{tabular}
\end{announce}
\begin{abstract}
本文简要介绍了博(硕)士毕业论文以及博士后工作报告模板 \verb|SISTThesis|的使用方法。

\keywords{博士,毕业论文, 博士后,工作报告, \LaTeX 模板.}
\end{abstract}

\begin{englishabstract}
This paper introduce the usage of the \LaTeX document template class \verb|SISTThesis|.  \verb|SISTThesis| provide the templates of thesis for doctor (master) degree, and postdoctoral research report.

\englishkeywords{\LaTeX\ Templates, thesis, postdoctoral research report.}

\end{englishabstract}

\tableofcontents
\listoffigures
\listoftables

\chapter{符号和约定}
\subsection*{本报告对符号的使用做如下约定:}
\begin{tabular}{p{5cm}p{9cm}}
    $\bm{u}$ & 加粗斜体小写字母表示向量\\
    %$\bmu{u} $ &  带下划线的加粗斜体小写字母表示分块向量\\
    $\bm{K}$ & 加粗斜体大写字母表示矩阵\\
    %$\bmu{K}$ & 带下划线的加粗斜体大写字母表示分块矩阵\\
    $a,b,c$ & 斜体小写字母表示数字\\
    $\mathcal{C},\mathcal{I} , \mathcal{F}$ &欧拉书写体表示算子或者$\sigma$-代数
\end{tabular}


\subsection*{本报告中符号的含义:}
\begin{tabular}{p{5cm}p{9cm}}
    $\mu$ & 非负测度 \\
    $\rho$ & 概率密度函数或者权函数\\
    $D$ & 物理空间上的区域\\
    $\Gamma$ & 随机变量空间上的区域\\
    $\langle \cdot, \cdot \rangle_{\rho}$ & 表示以$\rho$为权函数的内积\\
    $\Vert\cdot\Vert_{\rho}$ & 内积$\langle \cdot, \cdot \rangle_{\rho}$的诱导范数\\
    $\mathbb{E}[\cdot]$  & 期望\\
    $\mathbb{V}[\cdot]$ & 方差\\
    $\delta_{i,j}$ & delta 函数\\
    $\otimes$ & Kronecker积\\
    $\mathrm{Cov}_a(\bm{x}_1,\bm{x}_2)$& 随机过程$a(\bm{x},\omega)$的协方差函数\\
    
\end{tabular}



%%正文部分
\mainmatter
\chapter{引言}
\texttt{SISTThesis}~宏包的编写缘于笔者需要撰写博士后工作报告。尽管学校/研究所提供了Word格式的工作报告模板,但由于报告中含有大量的数学公式,Word无法满足要求,所以只能寻找\LaTeX 格式的报告模板。笔者在CTEX下载中心找到了吴凌云撰写的宏包\texttt{PostDocRep},但可能是由于版本的原因(撰写于2009年),尝试多种编译方式都无法通过编译。于是只好自己尝试着编写了一个模板。由于毕业论文和博士后工作报告的主要差别是封面等,索性将其扩展,使其包含了学位论文的格式。

\texttt{SISTThesis}~宏包主要根据博士后研究工作报告的撰写要求编写,目的是简化博士后研究工作报告的撰写,使作者可以将精力集中到报告的内容上而不是浪费在版面设置上。同时,宏包还参考了其它学校的\LaTeX 论文模板,添加了博(硕)士学位论文的选项(尚未找到上海科技大学的博/硕士学位论文要求)。

因为工作报告参考文献使用GB7714标准,因此文档中还包含了Zeping Lee(李泽平) 编写的参考文献的样式文件\verb|gbt-7714-2015-author-year.bst|(参考文献按照作者-日期排序)以及\verb|gbt-7714-2015-numerical.bst|(参考文献按照引用顺序排序)。
\section{系统要求}
SISTThesis 宏包通过 ctex 宏包来获得中文支持。 ctex 宏包提供了一个
统一的中文\LaTeX 文档框架, 它可以从\url{http://www.ctex.org}网站下载。

此外,SISTThesis 宏包还使用了宏包 amsmath、 amsthm、 amsfonts、
amssymb、 bm 和 hyperref。
\section{下载与安装}
SISTThesis宏包的最新版本可以从\url{https://github.com/sslchi/SISTThesis}下载。

宏包包含一个文档类型文件SISTTheis.cls以及一个使用模板文件main.tex(也即本文的源文件), 用户
可以通过修改这个模板来编写自己的学位论文。
\section{问题反馈}
用户在使用中遇到问题, 可以联系作者:
\begin{center}
      sslchi@126.com
\end{center}
\chapter{基本选项及命令介绍}\label{sec:basic}
本章介绍模板的一些基本选项和命令.

\section{编译}
本模板基于cetxbook宏包,暂时只支持xeLaTeX方式进行编译. 使用{\bfseries TexStudio编辑器}的用户可以直接点击编辑器上的构建并查看按钮进行编译,无需进行设置; 使用其它编辑器的用户请设置编译方式为xeLaTeX.

此外, 本文档中的编码方式皆为UTF-8, 若用编辑器打开之后显示乱码,请将编辑器编码方式设置为UTF-8(强烈安利TexStudio).


\section{博士后工作报告中的选项及命令}
\begin{tabular}{p{5cm}p{9cm}}
   \verb|\title|         & 标题\\
   \verb|\author|        & 作者\\
   \verb|\jobbegin|      & 工作开始时间\\
   \verb|\jobend|        & 工作期满时间\\
   \verb|\reportfinish|  & 工作完成日期\\ 
   \verb|\submitdate|    & 报告提交日期\\
   \verb|\englishtitle|  & 英文标题\\
   \verb|\discipline|    & 一级学科\\
   \verb|\major|         & 二级学科\\
   \verb|\institute|     & 研究所\\
   \verb|\address|       & 研究所所在地\\
\end{tabular}
\section{学位论文中的选项及命令}

\chapter{数学公式}
参见刘海洋~\cite{Liu2013}
\chapter{表格图形}\label{sec:tabfig}

\section{表格}
\begin{table}[htbp]
    \caption{表格示例}\label{tab:dem}
    \begin{center}
       \begin{tabular}{|ccclll|}
        \hline
       1 & 2 & 3 & 4 & 5 & 6\\
       \hline
       1 & 2 & 3 & 4 & 5 & 6\\
       \hline
       \end{tabular}
    \end{center}
\end{table}
\section{图形}
\begin{figure}[htbp]
    \begin{center}
        \includegraphics[width=0.3\linewidth]{logo.jpg}
    \end{center}
    \caption[图形示例]{校徽}
\end{figure}
\appendix
\include{chap/req}


\backmatter
\begin{thanks}

\vskip 18pt

我要感谢......


\end{thanks}
\bibliography{book,journal}
\addcontentsline{toc}{chapter}{\bibname}

\begin{resume}

\begin{resumesection}{基本情况}
xxx, 男, 中国科学院上海微系统与信息技术研究所博士后。
\end{resumesection}


\begin{resumelist}{教育状况}
2xxx 年~9 月至~2xxx 年~7
月,xx大学xx系,研究生,专业:xxxx。
 
xxxx 年~9 月至~xxxx 年~7 月,xx大学xxx学院,
本科,专业:xxxx。
\end{resumelist}

\begin{resumelist}{工作经历}
 2xxx 年~7 月至~2xxx 年~3
 月,中国科学院上海微系统与信息技术研究所,博士后。
\end{resumelist}

\begin{resumelist}{研究兴趣}
xxx,xxx。
\end{resumelist}

\begin{resumelist}{联系方式}

E-mail: xxxx@126.com
\end{resumelist}

\end{resume}
\begin{publications}{99}
\item {\bf sslchi}. 
``\LaTeX\ Template of Postdoctoral Research Report",  {\em Journal of SIST}, {\bf 57}, 171-181, 2018.
\item {\bf sslchi}. 
``\LaTeX\ Template for SIST of ShanghaiTech University",  {\em Journal of ShanghaiTech Universiy}, {\bf 57}, 171-181, 2018.
\end{publications}


\end{document}
